\documentclass{report}
\usepackage[brazil]{babel}
\usepackage[utf8]{inputenc}
\usepackage[num]{abntcite} % Carregamos o pacote abntcite com a opção alf, ou seja, citações alfanuméricas
\usepackage{listings}
\usepackage{url}
%\RequirePackage[colorlinks=false,hyperindex,plainpages=false]{hyperref}


\lstset{language=C,
	extendedchars=true,
	inputencoding=utf8,
	showstringspaces=false,
	tabsize=2,
	texcl=true,
	basicstyle=\scriptsize,
	escapechar={\@},
	breaklines=true
}


\author{Cristiano Medeiros Dalbem \and Fábio da Fontoura Beltrão \and Lucas Fialho Zawacki}
\title{GRASP Aplicado ao Problema de Balanceamento de Linhas de Produção}
%\date{}

\begin{document}

\maketitle
\tableofcontents

\chapter{Introdução}

\chapter{Definição do Problema}

Tirada diretamente da definição do trabalho.

Uma linha de produção consiste em uma série de estações de trabalho. Dado um conjunto de tarefas com restrições de
precedência, temos que atribuir as tarefas às estações, tal que a precedência é respeitada. Cada tarefa possui um tempo
de execução. O tempo total das tarefas de uma estação define a sua \emph{carga}, e a carga máxima entre todas as
estações define o \emph{tempo de ciclo}. Formalmente

\begin{description}
 \item [Instância] Um grafo G = (T,P) direcionado acíclico sobre um conjunto de tarefas T, o tempo de execução $t_i$ de
cada tarefa $i \in T$, e um número $m$ de estações de trabalho.

 \item [Solução] Uma atribuição $s : T \rightarrow [m]$ das tarefas às estações que satisfaz as restrições de
precedência, i.e. para cada $tu \in P$, $s(t) \leq s(u)$.

 \item [Objetivo] Minimizar o tempo de ciclo  $max_{i\in[m]}\sum_{j\in T\mid s(t)=i}t_j$.
\end{description}




\chapter{GRASP}

\section{Solução Incial: Construção Gulosa Randomizada}

Não consideramos nossa abordagem muito condizente com aquilo que estudamos sobre GRASP. Mas devido à natureza do
problema, de longe pareceu a abordagem mais eficiente, ambos em termos de eficiência computacional (neste caso medindo
somente tempo de execução) como eficiência na qualidade da resposta.

\section{Busca Local}

\chapter{Resultados}

\chapter{Conclusão}

\end{document}
