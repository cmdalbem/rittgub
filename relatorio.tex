\documentclass{report}
\usepackage[brazil]{babel}
\usepackage[utf8]{inputenc}
\usepackage[num]{abntcite} % Carregamos o pacote abntcite com a opção alf, ou seja, citações alfanuméricas
\usepackage{listings}
\usepackage{hyperref}
\usepackage{url}
%\RequirePackage[colorlinks=false,hyperindex,plainpages=false]{hyperref}


\lstset{language=C,
	extendedchars=true,
	inputencoding=utf8,
	showstringspaces=false,
	tabsize=2,
	texcl=true,
	basicstyle=\scriptsize,
	escapechar={\@},
	breaklines=true
}


\title{GRASP Aplicado ao Problema de Balanceamento de Linhas de Produção}
\author{Cristiano Medeiros Dalbem \and Fábio da Fontoura Beltrão \and Lucas Fialho Zawacki\\
\\
\\
\small Otimização Combinatória (INF05010) - Prof. Marcus Ritt\\
\small Instituto de Informática\\[-0.8ex]
\small Universidade Federal do Rio Grande do Sul
}
%\date{}

\begin{document}

\maketitle
\tableofcontents

\chapter{Introdução}

O trabalho final da disciplina de Otimização Combinatória consiste na escolha de
 um problema de otimização e de uma meta-heurística, e o objetivo é implementar
 uma solução para aquele utilizando-se deste. O problema escolhido foi o
 Balanceamento de linhas de produção \cite{salbp}, e a meta-heurística a GRASP \cite{grasp}.

GRASP é um acrônimo para \emph{Greedy Randomized Adaptive Search Procedure},
cujo nome já especifica muito bem a sua filosofia. O objetivo é formular o
programa com uma heurística gulosa, e o procedimento de resolução da
meta-heurística randomizará cada passo com uma das soluções "mais gulosas".
Para cada solução criada é feita uma Busca Local (\emph{best improvement} ou
\emph{first improvement}, dependendo da implementação),
e o processo é repetido por um número pré-determinado de iterações.

Como objetivos adicionais do trabalho estão a formulação do problema como
um problema de programação inteira/linear, e a execução das soluções
desenvolvidas (GLPK e GRASP) sobre um conjunto de casos de
teste providos pelo professor.

\chapter{Definição do Problema}

Tirada diretamente da definição do trabalho.

Uma linha de produção consiste em uma série de estações de trabalho.
Dado um conjunto de tarefas com restrições de
precedência, temos que atribuir as tarefas às estações, tal que a precedência
é respeitada. Cada tarefa possui um tempo
de execução. O tempo total das tarefas de uma estação define a sua
\emph{carga}, e a carga máxima entre todas as
estações define o \emph{tempo de ciclo}. Formalmente

\begin{description}
 \item [Instância] Um grafo G = (T,P) direcionado acíclico sobre um conjunto de
 tarefas T, o tempo de execução $t_i$ de
cada tarefa $i \in T$, e um número $m$ de estações de trabalho.

 \item [Solução] Uma atribuição $s : T \rightarrow [m]$ das tarefas às
 estações que satisfaz as restrições de
precedência, i.e. para cada $tu \in P$, $s(t) \leq s(u)$.

 \item [Objetivo] Minimizar o tempo de ciclo  $max_{i\in[m]}\sum_{j\in T\mid s(t)=i}t_j$.
\end{description}


\chapter{Programação inteira}

\section{Formulação inteira}

\begin{center}
\begin{tabular}{ r l }
  \textbf{min} & x \\
  \textbf{s.a} & $a = 1$ \\
  & $b = 2$ \\
  & $x = INF$ \\
\end{tabular}
\end{center}

\section{Solução em GLPK}

\begin{lstlisting}
set prods;

param custoInicial{prods} >= 0;
param preco{prods} >= 0;

var quantidade{ prods } >= 0, integer;
var produz{prods } binary;

maximize lucro:
	sum {j in prods} (preco[j]*quantidade[j] - custoInicial[j]*produz[j]);

subject to limiteProducao:
	(sum {j in prods} produz[j]) <= 2;

subject to limiteQuantidade:
	(produz['1'] + produz['2'])*2 >= produz['3'] + produz['4'];

subject to precisaProduzir {j in prods}:
	quantidade[j] <= 2000*produz[j];


data;

param : prods : custoInicial preco :=
	'1'	50000		70
	'2'	40000		60
	'3'	70000		90
	'4'	60000		80;

end;
\end{lstlisting}

\chapter{GRASP}

O grupo tentou duas abordagens para a resolver o problema. Uma delas se encaixa
perfeitamente no modelo do GRASP,  a outra nem tanto.

\section{Primeira Abordagem}

\subsection{Construção Gulosa Randomizada}

Para cada tarefa do vetor de tarefas o algoritmo tenta colocá-la, se
isto respeitar o grafo de precedência, em alguma
máquina de maneira que o maior ciclo entre todas as máquinas
seja o menor possível. Cada um desses resultados (o valor de um maior ciclo)
é colocado de maneira ordenada em uma Restricted Candidate List (\emph{RCL})
de tamanho $n$ de tal sorte que se for adicionado um elemento quando ela estiver
cheia o último seja descartado. Após construir a \emph{RCL} retiramos um dos
elementos de maneira aleatória e esta será a máquina na qual deveremos inserir
a tarefa atual.

O processo é repetido para cada uma das tarefas e quando ele acaba teremos
uma atribuição de tarefas $i$ para cada uma das $j$ máquinas.

\section{Busca Local}

Após gerada a solução incial, o algoritmo parte para a busca local.
O objetivo da busca local é ``caminhar'' na
vizinhança da solução, tentando sempre melhorá-la, até não conseguir mais.
Nossa vizinhança é: sabendo que a máquina $m*$
é a máquina que determina o \emph{tempo de ciclo},
tentamos retirar uma das tarefas $t*$ atribuida a essa máquina e, para cada
outra máquina $m$, se for possível atribuir $t*$ à $m$, isto é feito.
Utilizamos \textit{first improvement} na nossa busca
local, fazendo com que o primero vizinho que melhore a solução seja
escolhido como próxima solução.

Há, porém, o problema de existir mais que uma máquina $m*$,
definidora do \emph{tempo de ciclo}. Tendo isto em vista,
durante a busca local, é considerado que o vizinho melhorou a
solução se o \emph{tempo de ciclo} daquele vizinho for
menor ou se o \emph{tempo de ciclo} do vizinho for igual mas a
quantidade de máquinas $m*$, que definem este tempo, for
menor. Isto evita que a busca local entre em \emph{loop}, mas não impede que
eventualmente ela melhore as \emph{cargas} de
todas máquinas $m*$, melhorando o \emph{tempo de ciclo} global.

\subsection{Resultados e Limitações}

Esta abordagem, principalmente a etapa de construção gulosa, não se mostrou muito
efetiva e os resultados deixaram a desejar. A seguir uma lista com os resultados:

Todos resultados foram obtidos utilizando como limite 10000 iterações.

\begin{table}[htbp]
 \begin{tabular}{|c|c|c|c|c|}
  \hline
  \textbf{Instância} & \textbf{m} & \textbf{Solução Ótima} & \textbf{Solução Obtida} & \% optimalidade \\
  \hline
  Arcus2 & 27 & 5689 & 6125 & 92.88 \\
  \hline
  Arcus2 & 9 & 16711 & 16839 & 99.23 \\
  \hline
  Bathol2 & 51 & 84 & 95 & 88.42 \\
  \hline
  Bathol2 & 27 & 157 & 166 & 94.57 \\
  \hline
  Warnecke & 10 & 155 & 160 & 96.87 \\
  \hline
  Warnecke & 20 & 79 & 85 & 92.94 \\
  \hline
  Scholl & 50 & 1394 & 1506 & 92.56 \\
  \hline
  Scholl & 38 & 1834 & 1938 & 94.63 \\
  \hline
  Scholl & 25 & 2787 & 2862 & 97.37 \\
  \hline
  Wee-Mag & 20 & 77 & 81 & 95.06 \\
  \hline
  Wee-Mag & 30 & 56 & 57 & 98.24 \\
  \hline
 \end{tabular}
\end{table}

O grupo acredita que a ineficácia do algoritmo se deva a uma tendenciosidade do
processo de construção de uma solução. Uma solução em que uma tarefa que denota
muitas dependências fica designada para uma máquina $i$, tal que $i$ é muito
próximo de $n$, acaba sendo bastante ruim. Nesses casos a busca local não
conseguiu explorar o espaço de soluções e acabou com uma solução muito aquém
do desejado.

\section{Segunda Abordagem}

Visando melhorar os resultados o grupo pensou em uma segunda abordagem para a
etapa de construção gulosa e esta, talvez devido à natureza do problema,
de longe se mostrou  mais bem sucedida, em termos de
eficiência computacional (neste caso medindo
somente tempo de execução) e qualidade da resposta.

Primeiramente as tarefas são ordenadas topologicamente.
Um problema aqui é que a ordenação é deterministica e gera
sempre o mesmo resultado, e isso é indesejado. Por isso, sempre que temos mais
de uma tarefa possível para ir na
posição $i$ do vetor ordenado, escolhemos randomicamente entre todas.

Tendo o vetor ordenado, calculamos um limite inferior para a
resposta. Isto é feito pegando o valor máximo entre $\frac{\sum_{i\in T}t_i}{m}$
 e $max_{i\in T}t_i$. Sendo o primeiro
o valor da distribuição mais uniforme possível, ou seja,
soma dos custos de todas tarefas e dividido pelo
número de máquinas e o segundo o custo da tarefa mais custosa.

Sabendo esse limite inferior, procedemos: para cada máquina, coloca-se
nela uma das tarefas que em ordem topológica pode ser atribuída.
Isto é feito enquanto a \emph{carga}
da máquina for menor que o limite inferior
calculado. Quando este valor for extrapolado, é preciso decidir se iremos
adicionar esta tarefa a mais
(extrapolando o limiar) ou se vamos manter como está (ficando abaixo do limiar).
Sempre é tomada a decisão que mais
aproxima a \emph{carga} da máquina do limiar calculado.

Em resumo, o que ele faz é tentar aproximar a \emph{carga} de
cada máquina do limite inferior calculado.
Como isso não garante a atribuição de todas tarefas, isso é forçado, no fim,
colocando todas tarefas pendentes na última máquina.

A etapa de busca local permanece igual à anterior.

\chapter{Resultados}

botar tabela

\chapter{Conclusão}

Os resultados obtidos usando a segunda abordagem se mostraram bastante
satisfatórios, tendo em média uma porcentagem de optimalidade de 94.79. O
GRASP se mostrou como uma heurística muito boa, não só pelos bons resultados,
mas também pela facilidade em se
trabalhar.

De fato, de um modo geral, é bastante fácil entender e implementar o GRASP:
os conceitos envolvidos são bastante simples e ele é composto apenas de duas
funções principais (construção gulosa e busca local) além de um parâmetro
ajustável de randomicidade. A dificuldade  que o grupo teve foi a parte
dependente do problema sendo tratado, foi difícil formular uma vizinhança e
pensar em um algoritmo guloso que gerasse uma solução inicial válida.

\thebibliography{9}

\bibitem{salbp}
	\url{http://www.assembly-line-balancing.de/index.php?content=classview&content2=classview&content3=classviewdlfree&content4=classview&classID=28&type=dl}
	\emph{Homepage for Assembly Line Optimization Research (Assembly Line Balancing and Sequencing)}

\end{document}

