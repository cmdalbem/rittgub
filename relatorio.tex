\documentclass{report}
\usepackage[brazil]{babel}
\usepackage[utf8]{inputenc}
\usepackage[num]{abntcite} % Carregamos o pacote abntcite com a opção alf, ou seja, citações alfanuméricas
\usepackage{listings}
\usepackage{hyperref}
\usepackage{url}
%\RequirePackage[colorlinks=false,hyperindex,plainpages=false]{hyperref}


\lstset{language=C,
	extendedchars=true,
	inputencoding=utf8,
	showstringspaces=false,
	tabsize=2,
	texcl=true,
	basicstyle=\scriptsize,
	escapechar={\@},
	breaklines=true
}


\title{GRASP Aplicado ao Problema de Balanceamento de Linhas de Produção}
\author{Cristiano Medeiros Dalbem \and Fábio da Fontoura Beltrão \and Lucas Fialho Zawacki\\
\\
\\
\small Otimização Combinatória (INF05010) - Prof. Marcus Ritt\\
\small Instituto de Informática\\[-0.8ex]
\small Universidade Federal do Rio Grande do Sul
}
%\date{}

\begin{document}

\maketitle
\tableofcontents

\chapter{Introdução}

O trabalho final da disciplina de Otimização Combinatória consiste na escolha de um problema de otimização e de uma meta-heurística, e o objetivo é implementar uma solução para aquele utilizando-se deste. O problema escolhido foi o Balanceamento de linhas de produção \cite{salbp}, e a meta-heurística a GRASP \cite{grasp}.

GRASP é um acrônimo para \emph{Greedy Randomized Adaptive Search Procedure}, cujo nome já especifica muito bem a sua filosofia. O objetivo é formular o programa com uma heurística gulosa, e o procedimento de resolução da meta-heurística randomizará cada passo com uma das soluções "mais gulosas". Para cada solução criada é feita uma Busca Local (Best Fit ou First Fit, dependendo da implementação), e o processo é repetido por um número pré-determinado de iterações.

Como objetivos adicionais do trabalho estão a formulação do problema como um problema de programação inteira/linear, e a execução das soluções desenvolvidas (GLPK e GRASP) sobre um conjunto de casos de teste providos pelo professor.

\chapter{Definição do Problema}

Tirada diretamente da definição do trabalho.

Uma linha de produção consiste em uma série de estações de trabalho. Dado um conjunto de tarefas com restrições de
precedência, temos que atribuir as tarefas às estações, tal que a precedência é respeitada. Cada tarefa possui um tempo
de execução. O tempo total das tarefas de uma estação define a sua \emph{carga}, e a carga máxima entre todas as
estações define o \emph{tempo de ciclo}. Formalmente

\begin{description}
 \item [Instância] Um grafo G = (T,P) direcionado acíclico sobre um conjunto de tarefas T, o tempo de execução $t_i$ de
cada tarefa $i \in T$, e um número $m$ de estações de trabalho.

 \item [Solução] Uma atribuição $s : T \rightarrow [m]$ das tarefas às estações que satisfaz as restrições de
precedência, i.e. para cada $tu \in P$, $s(t) \leq s(u)$.

 \item [Objetivo] Minimizar o tempo de ciclo  $max_{i\in[m]}\sum_{j\in T\mid s(t)=i}t_j$.
\end{description}


\chapter{Programação inteira}

\section{Formulação inteira}

\begin{center}
\begin{tabular}{ r l }
  \textbf{min} & x \\
  \textbf{s.a} & $a = 1$ \\
  & $b = 2$ \\
  & $x = INF$ \\
\end{tabular}
\end{center}

\section{Solução em GLPK}

\begin{lstlisting}
set prods;

param custoInicial{prods} >= 0;
param preco{prods} >= 0;

var quantidade{ prods } >= 0, integer;
var produz{prods } binary;

maximize lucro:
	sum {j in prods} (preco[j]*quantidade[j] - custoInicial[j]*produz[j]);

subject to limiteProducao:
	(sum {j in prods} produz[j]) <= 2;

subject to limiteQuantidade:
	(produz['1'] + produz['2'])*2 >= produz['3'] + produz['4'];

subject to precisaProduzir {j in prods}:
	quantidade[j] <= 2000*produz[j];


data;

param : prods : custoInicial preco :=
	'1'	50000		70
	'2'	40000		60
	'3'	70000		90
	'4'	60000		80;

end;
\end{lstlisting}




\chapter{GRASP}

\section{Solução Incial: Construção Gulosa Randomizada}

Não consideramos nossa abordagem muito condizente com aquilo que estudamos sobre GRASP. Mas devido à natureza do
problema, de longe pareceu a abordagem mais eficiente, ambos em termos de eficiência computacional (neste caso medindo
somente tempo de execução) como eficiência na qualidade da resposta.

\section{Busca Local}

\chapter{Resultados}

\chapter{Conclusão}


\thebibliography{9}

\bibitem{salbp}
	\url{http://www.assembly-line-balancing.de/index.php?content=classview&content2=classview&content3=classviewdlfree&content4=classview&classID=28&type=dl}
	\emph{Homepage for Assembly Line Optimization Research (Assembly Line Balancing and Sequencing)}

\end{document}
